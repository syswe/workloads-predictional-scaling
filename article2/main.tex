\documentclass[12pt,a4paper]{article}
\usepackage[utf8]{inputenc}
\usepackage[T1]{fontenc}
\usepackage[turkish]{babel}
\usepackage{geometry}
\usepackage{graphicx}
\usepackage{longtable}
\usepackage{booktabs}
\usepackage{array}
\usepackage{siunitx}
\usepackage{enumitem}
\geometry{left=25mm,right=25mm,top=30mm,bottom=30mm}
\setlength{\parindent}{0pt}
\setlength{\parskip}{6pt}

\title{Kubernetes Tabanlı Tahmine Dayalı Yatay Pod Otomatik Ölçeklendiricinin\\Yük Örüntüsü Farkındalığı ve Büyük Dil Modeli Entegrasyonu ile Geliştirilmesi}
\author{}
\date{Ekim 2025}

\begin{document}
\maketitle

\begin{abstract}
Kubernetes tabanlı konteyner orkestrasyon sistemlerinde reaktif ölçeklendirme yaklaşımları, ani yük değişimlerine geç tepki vermekte ve kaynak kullanımında verimsizliğe neden olmaktadır. Bu çalışma, yüksek lisans tezi kapsamında geliştirilen örüntü-farkındalıklı tahmine dayalı yatay pod otomatik ölçeklendirici (PHPA) çerçevesinin syswe ad alanına taşınması ve operasyonel kullanıma hazır hale getirilmesi sürecini belgelemektedir. Altı temel yük örüntüsü için matematiksel formülasyonlar geliştirilmiş ve 2 milyondan fazla veri noktası içeren kapsamlı bir veri seti oluşturulmuştur. Yedi CPU-optimize makine öğrenmesi modeli hiperparametre optimizasyonu ile eğitilmiş ve örüntü-spesifik model seçimi yaklaşımı ile evrensel modellere kıyasla %37,4 ortalama iyileştirme elde edilmiştir. Not: LLM tabanlı örüntü tanıma önceki çalışmanın bir parçasıdır ve bu makalenin deneysel kapsamına dahil değildir. Mevcut çalışmada GBDT, CatBoost, VAR ve XGBoost modelleri operatöre entegre edilmiş, çevrimiçi ve çevrimdışı kıyaslama senaryolarını otomatikleştiren laboratuvar altyapısı geliştirilmiştir. Deneysel sonuçlar, tahmine dayalı modellerin standart HPA'ya kıyasla ölçeklendirme kararlarını öne çektiğini göstermektedir.
\end{abstract}

\section{Giriş (Introduction)}

Bulut tabanlı mikro servis mimarilerinin yaygınlaşması ile birlikte, konteyner orkestrasyon platformları modern yazılım altyapısının temel bileşenleri haline gelmiştir. Kubernetes, konteyner yönetimi ve otomatik ölçeklendirme yetenekleri ile bu alanda öne çıkan açık kaynak platformdur. Kubernetes Horizontal Pod Autoscaler (HPA), CPU ve bellek kullanımı gibi anlık metriklere dayalı reaktif ölçeklendirme sağlamakta, ancak ani yük değişimlerine geç tepki vermesi nedeniyle performans düşüşleri ve kaynak israfına yol açabilmektedir.

Literatürde reaktif ölçeklendirmenin sınırlamalarını aşmak üzere tahmine dayalı yaklaşımlar önerilmiş; zaman serisi analizi, makine öğrenmesi ve derin öğrenme teknikleri ile gelecekteki kaynak ihtiyaçlarının önceden kestirimi hedeflenmiştir. Bununla birlikte, pek çok çalışmanın yük örüntülerinin çeşitliliğini yeterince dikkate almayan “tek model her senaryoya uyar” varsayımıyla sınırlı kaldığı görülmektedir.

Bu çalışma, yazarın yüksek lisans tezi kapsamında geliştirilen örüntü-farkındalıklı Predictive Horizontal Pod Autoscaler (PHPA) çerçevesinin syswe ad alanına taşınması ve üretim ortamına hazır hale getirilmesini konu almaktadır. Altı temel yük örüntüsü (mevsimsel, büyüyen, patlama, açık/kapalı, kaotik, basamaklı) için matematiksel formülasyonlar geliştirilmiş; 600 senaryoda 2 milyondan fazla veri noktası üretilmiş; yedi farklı CPU-optimize tahmin modeli için hiperparametre optimizasyonu uygulanmıştır. Örüntü-spesifik model seçimi yaklaşımı ile evrensel modellere kıyasla ortalama %37,4 iyileştirme sağlanmıştır. LLM tabanlı örüntü tanıma önceki çalışmanın bir parçası olup bu makalenin kapsamı dışındadır; burada odak, en iyi modellerin operatör entegrasyonu ve deneysel doğrulanmasıdır.

Mevcut projede, GBDT, CatBoost, VAR ve XGBoost modelleri Kubernetes operatörü ile bütünleşik çalışacak şekilde uygulanmış; Helm şablonları ve CRD şemaları güncellenmiş; çevrimiçi ve çevrimdışı kıyaslama senaryolarını uçtan uca otomatikleştiren laboratuvar (labs) altyapısı geliştirilmiştir. Deneysel sonuçlar, tahmine dayalı modellerin standart HPA'ya kıyasla ölçeklendirme kararlarını anlamlı ölçüde öne çektiğini ve kaynak kullanımında daha dengeli noktalar sunduğunu göstermektedir.

\subsection{Katkılar (Contributions)}

Bu makalenin başlıca katkıları şunlardır:

\begin{itemize}[noitemsep]
  \item Örüntü taksonomisi ve 2M+ veri noktası ile sistematik değerlendirme zemini,
  \item Örüntü-spesifik model seçimi ile ortalama %37,4 doğruluk iyileştirmesi,
  \item (Bağlam) Önceki çalışmadan LLM bulguları; bu makalede uygulanmamıştır,
  \item PHPA operator entegrasyonu: GBDT, CatBoost, VAR, XGBoost,
  \item Tekrarlanabilir laboratuvar: çevrimdışı/çevrimiçi kıyaslama ve otomatik raporlama.
\end{itemize}

\section{İlgili Çalışmalar (Related Work)}

Kubernetes HPA belgelendirmesi ve temel otomatik ölçeklendirme ilkeleri, metrik-tabanlı reaktif karar mekanizmaları ile sınırlıdır. Reaktif yaklaşımlar, gecikmeler ortaya çıktıktan sonra devreye girdikleri için ani yük sıçramalarında hedef Servis Düzeyi Anlaşması (SLA) göstergelerini korumakta yetersiz kalabilmektedir. Çeşitli araştırmalar, tahmine dayalı stratejilerin bu boşluğu doldurabileceğini göstermektedir: zaman serisi kestirimi ve öğrenme tabanlı modellerle geleceğe dönük replika ihtiyaçları öngörülerek ölçeklendirme kararları öne çekilebilir.

Zaman serisi alanında Box–Jenkins ve Exponential Smoothing gibi klasik yöntemlerin yanında, Holt–Winters mevsimsellik modelleri üretim sistemlerinde pratikte sıkça değerlendirilmektedir. Makine öğrenmesi tarafında ise ağaç-tabanlı gradient boosting yöntemleri (GBDT, XGBoost, CatBoost) düşük hesaplama maliyeti ve yüksek öngörü performansı nedeniyle kurumsal ortamlarda yaygın olarak benimsenmektedir. Çok değişkenli zaman serisi için VAR gibi istatistiksel yöntemler, ek bağlamsal metrikler (CPU, bellek, ağ) ile beslendiğinde güçlü performans potansiyeli taşımaktadır.

Son dönemde, büyük dil modellerinin (LLM) zaman serisi ve örüntü tanıma problemlerine uygulanması gündeme gelmiştir. Zaman serisi yeniden-işleme (reprogramming), sıfır-atış (zero-shot) kestirim ve çok modlu (metin + görsel) analiz yaklaşımları ile LLM’ler geleneksel yöntemlere tamamlayıcı kabiliyetler eklemektedir. Bu çalışmada, LLM’ler yük örüntüsünün otomatik saptanması ve uygun tahmin modelinin önerimi için kullanılmış, böylece operasyonel iş akışına akıllı seçim katmanı eklenmiştir.

Özetle literatür, proaktif ölçeklendirmenin etkinliğini desteklerken, tek modelin tüm örüntülere “en iyi” çözüm olamayacağını, örüntü-spesifik seçimin kritik önem taşıdığını ortaya koymaktadır. Bu makale, altı temel örüntü için matematiksel formülasyonlar ve model eşleştirmesi sunarak; ayrıca LLM entegrasyonu ile otomatik örüntü tanıma sağlayarak literatüre bütüncül ve tekrarlanabilir bir çerçeve katkısı yapmaktadır.


\section{Sistem Mimarisi (System Architecture)}

PHPA çerçevesi; örüntü üretimi ve veri hazırlama, model eğitimi ve seçimi, LLM tabanlı örüntü tanıma ve operatör entegrasyonundan oluşan uçtan uca bir mimari sunar. Veri akışı şu adımlarla ilerler: (i) Sentetik örüntü verisi oluşturulur ve gerçek dünya davranışlarını yansıtacak biçimde kalibre edilir; (ii) CPU-optimize tahmin modelleri, örüntü-spesifik hiperparametre optimizasyonu ile eğitilir; (iii) LLM katmanı, CSV ve grafikten örüntü sınıfını otomatik saptar ve uygun aday modelleri önerir; (iv) Operatör, CRD ve algoritma koşucuları (runner) ile kestirimleri ölçeklendirme kararına entegre eder.

Bu mimari, hem çevrimdışı araştırma (benchmark) döngülerini hem de gerçek Kubernetes kümesinde çevrimiçi deneyleri destekler. Projede sunulan laboratuvar (labs) altyapısı; bootstrap, offline/online deney, raporlama ve temizlik adımlarını tek komutla otomatikleştirir. Böylece hem akademik değerlendirme hem de pratik doğrulama senaryoları tekrarlanabilir şekilde icra edilir.

Operasyonel açıdan, PHPA operatörü Kubernetes HPA davranışları ile uyumlu olacak şekilde çalışır; `syncPeriod` üzerinden periyotlandırılmış bir döngü ile mevcut metriklerden yalın HPA hedefi üretir, ardından bir veya birden fazla tahmin modeli ile ileriye dönük replika tahmini yaparak “karar tipi” (ör. maksimum, ortalama, medyan) stratejilerine göre nihai ölçeklendirme kararını verir.


\section{Veri Seti ve Örüntüler (Dataset and Patterns)}

Araştırmanın temelini oluşturan örüntü taksonomisi, gerçek dünya iş yüklerinin geniş bir yelpazesini kapsayacak şekilde tasarlanmıştır. Altı temel örüntü sınıfı için matematiksel formülasyonlar belirlenmiş ve 35 günlük, 15 dakikalık örnekleme aralığında senaryolar üretilmiştir. Bu kapsamda 600 senaryo boyunca 2 milyondan fazla zaman serisi veri noktası elde edilmiştir.

\subsection{Matematiksel Formülasyonlar}

\textbf{Mevsimsel (Seasonal)}
\begin{equation}
P_t = B + \sum_k A_k \sin\left(\frac{2\pi t}{T_k} + \phi_k\right) + N_t
\end{equation}

\textbf{Büyüyen (Growing)}
\begin{equation}
P_t = B + G\cdot f(t) + S \cdot \sin\left(\frac{2\pi h_t}{24}\right) + N_t
\end{equation}

\textbf{Patlama (Burst)}
\begin{equation}
P_t = B + \sum_i B_i\, g(t - t_i, d_i)\, \mathbf{1}_{t_i \le t < t_i + d_i} + N_t
\end{equation}

\textbf{Açık/Kapalı (On/Off)}
\begin{equation}
P_t = \begin{cases}
P_{high} + N_t^{high} & \text{eğer } S_t = 1 \\
P_{low} + N_t^{low} & \text{eğer } S_t = 0
\end{cases}
\end{equation}

\textbf{Kaotik (Chaotic)}: Çok bileşenli, düzensiz ve öngörülmesi güç süreçlerin bir bileşimi olarak modellenir.

\textbf{Basamaklı (Stepped)}
\begin{equation}
P_t = B_{base} + L_t \cdot S_{step} + S \cdot \sin\left(\frac{2\pi h_t}{24}\right) + N_t
\end{equation}

\subsection{Kalibrasyon ve Gerçekçilik}

Örüntü parametreleri, gerçek dünya kaynak kullanım günlükleri (ör. kurumsal web servisleri, spor etkinlikleri sırasında dalgalanan trafik, NASA web günlükleri gibi literatürde rastlanan aşırı yük durumları) göz önünde bulundurularak kalibre edilmiştir. Tek değişkenli senaryolar, operatörün çalışma modeline paralel olacak şekilde pod sayısı üzerinden kurgulanmış; çok değişkenli potansiyeli göstermek üzere zaman bileşenleri ve türetik özellikler de veri setine eklenmiştir. Bu yaklaşım, hem istatistiksel değerlendirme hem de operasyonel entegrasyon açısından dengeli bir temel sunmaktadır.


\section{Modeller ve Optimizasyon (Models and Optimization)}

Bu çalışmada yedi model ailesi değerlendirilmiştir: Linear, Holt–Winters, GBDT, XGBoost, CatBoost, VAR ve Prophet. Operasyonel entegrasyon kapsamında GBDT, XGBoost, CatBoost ve VAR modelleri operatör tarafından çalıştırılabilir hale getirilmiştir. Modellerin seçiminde temel hedefler; (i) CPU üstüne optimize düşük gecikmeli kestirim, (ii) sınırlı hafıza tüketimi, (iii) örüntüye göre uyarlanabilir hiperparametreler ve (iv) operator tarafında hızlı ardışık tahmin yeteneğidir.

\subsection{Özellik (Feature) Tasarımı}

Tek değişkenli zaman serileri için lag tabanlı özellikler (örn. son $k$ adım), hareketli ortalama/medyan pencereleri ve zaman tabanlı bileşenler (saat, gün, haftanın günü) kullanılmıştır. Amaç; ağaç tabanlı modellerin güçlü olduğu ayrım sınırlarını zenginleştirmek ve VAR gibi yöntemler için çok değişkenli bağlam oluşturmaktır.

\subsection{Hiperparametre Optimizasyonu}

Model hiperparametreleri örüntü-spesifik olarak ayarlanmış, zamanın yönünü koruyan temporal çapraz doğrulama stratejileri ile doğrulanmış ve erken durdurma kriterleri ile aşırı öğrenme engellenmiştir. Böylece hem kestirim isabeti hem de çalışma süresi arasında pratik denge elde edilmiştir.

\subsection{Model Aileleri — Özet}

\textbf{Linear:} Basit ve hızlıdır; kritik durumlarda agresif erken tepki için tercih edilebilir. \texttt{lookAhead} ve sınırlı \texttt{historySize} ile düşük hesaplama maliyeti sağlar.

\textbf{Holt–Winters:} Mevsimsellik ve trend için uygundur. \textit{alpha, beta, gamma} ile düzey, eğilim ve mevsimsellik yumuşatma katsayıları ayarlanır; \texttt{seasonalPeriods} ve \texttt{storedSeasons} ile dönem uzunluğu ve hafıza sınırı kontrol edilir.

\textbf{GBDT/XGBoost:} Ağaç-tabanlı gradient boosting ailesi, zengin özellik seti ile kuvvetli performans ve hız sunar. Temporal CV ve erken durdurma ile aşırı öğrenme engellenir; \texttt{lags} ve \texttt{lookAhead} işletim anında etkili parametrelerdir.

\textbf{CatBoost:} Kategorik özelliklerle güçlüdür. Tek değişkenli sentetik veride sınırlı görünse de gerçek dünyada türetik/kategorik özniteliklerle performansı yükselir. Operatör tarafında iteratif ileri bakışla hızlı çıkarım yapılır.

\textbf{VAR:} Çok değişkenli zaman serileri için uygundur. Zaman bileşenleri ve diğer metriklerle beslendiğinde anlamlı kestirim yapar; tek değişkenli senaryolarda kararsızlık gösterebilir.

\textbf{Prophet:} Araştırma sürecinde referans amaçlı değerlendirilmiştir; operatör entegrasyonu kapsamda değildir.

\subsection{Örüntü-Spesifik Model Seçimi}

Yüksek lisans tezinde gösterildiği üzere, tek bir evrensel model yerine örüntü-spesifik seçim ortalama %37,4 MAE iyileştirmesi sağlamaktadır. Bu çalışmada da aynı ilke korunmuş ve operatör seviyesinde LLM tabanlı otomatik örüntü tanıma ile aday modellerin önerimi birleştirilmiştir. Kritik servisler için hızlı tepki (Linear/GBDT), dengeli maliyet-performans (XGBoost/CatBoost), mevsimsellik (Holt–Winters) ve çok değişkenli bağlam (VAR) ilkeleriyle seçim stratejileri tanımlanmıştır.

\section{Operatör ve Uygulama (Operator Implementation)}

PHPA, Kubernetes üzerinde bir Operatör olarak çalışır. Temel yapı taşları; CRD (CustomResourceDefinition), denetleyici (controller), model geçmişi depolaması ve algoritma koşucu (runner) katmanıdır. Operatör; mevcut metriklerle klasik HPA hedefini hesaplar, ardından bir veya daha fazla tahmin modelini çalıştırarak ileriye dönük replika değerlerini üretir ve seçilen karar tipine göre (maksimum, minimum, ortalama, medyan) nihai ölçeklendirme kararını verir.

\subsection{CRD ve API}

`PredictiveHorizontalPodAutoscaler` (v1alpha1) kaynak türü; ölçeklenecek hedefi (`scaleTargetRef`), min/max replika sınırlarını, kullanılacak metrikleri, senkronizasyon periyodunu (`syncPeriod`) ve \texttt{models} listesi üzerinden model yapılandırmalarını içerir. Model türleri (Linear, HoltWinters, GBDT, XGBoost, VAR, CatBoost) enum ile sınırlandırılmıştır. Her model için veri geçmişi boyutu (`historySize`), bakılacak gecikme sayısı (`lags`) ve ileriye bakış süresi (`lookAhead`) gibi ayarlar sağlanır. Holt–Winters için \textit{alpha, beta, gamma}, mevsimsellik ve sönümlü trend gibi ileri seçenekler ile HTTP tabanlı \textit{runtime tuning hook} desteği bulunmaktadır.

\subsection{Algoritma Koşucu (Runner)}

Operatör çalışma zamanında, JSON iletilen \texttt{replicaHistory} ve model parametrelerini hafif Python uygulamalarına aktarır (ör. \texttt{algorithms/catboost/catboost.py}). Bu betikler; lag özelliklerini üretir, tek adım ya da iteratif ileri bakış ile tamsayı bir replika tahmini üretir ve standard output üzerinden geri döndürür. Bir \texttt{calculationTimeout} süresi içinde cevap üretilemezse hesaplama atlanır.

\subsection{Ölçek Davranışı ve Stratejiler}

Operatör; HPA ile eşdeğer yukarı-aşağı ölçekleme politikalarını, stabilizasyon pencerelerini ve olay geçmişi kaydını korur. Birden fazla modelin kullanıldığı senaryolarda \texttt{decisionType} alanı ile maksimum, minimum, ortalama ya da medyan üzerinde karar verilir. Bu yaklaşım, farklı modellerin güçlü yönlerini birleştirme esnekliği sağlamaktadır.

\subsection{Model Geçmişi ve Kalıcılık}

Model geçmişi depolaması, \texttt{PredictiveHorizontalPodAutoscalerData} yapısı ile bir ConfigMap içinde sürdürülür; her model için \texttt{replicaHistory} dizisi ve \texttt{syncPeriodsPassed} bilgisi tutulur. Bu sayede model geçmişi yeniden başlatmalar veya seyrek hesaplama döngüleri boyunca korunur; \texttt{perSyncPeriod} ile uyumlu şekilde modelin hesaplama sıklığı denetlenir.

\subsection{Koşucu ve Genişletilebilirlik}

Çalışma zamanı koşucu arayüzü, komut çağrısı ve çıktı ayrıştırma sorumluluklarını izole ederek operatöre yeni modeller eklemeyi kolaylaştırır. Örneğin CatBoost (\texttt{algorithms/catboost/catboost.py}), Holt–Winters (\texttt{algorithms/holt\_winters/holt\_winters.py}) ve VAR (\texttt{algorithms/var/var.py}) için aynı şablon korunur: girdide \texttt{replicaHistory}, çıktıda tek tamsayı replika kestirimi.

\subsection{Doğrulama ve Ölçek Davranışı}

Model yapılandırmaları \texttt{validation} katmanında doğrulanır; eksik alanlar veya tip uyuşmazlıkları erken aşamada raporlanır. \texttt{scalebehavior} mantığı yukarı/aşağı ölçekleme politikaları ile stabilizasyon pencerelerini uygular; \texttt{decisionType} ise birden fazla model çıktısında nihai kararın nasıl alınacağını belirler (maksimum, minimum, ortalama, medyan).

\section{Deney Düzeneği (Experimental Setup)}

Deneyler, proje kapsamında sunulan \texttt{labs} altyapısı ile otomatikleştirilmiştir. Bu altyapı; kurulum (bootstrap), çevrimdışı kıyaslama (offline), çevrimiçi kıyaslama (online), raporlama (report) ve temizlik (destroy) adımlarını \texttt{make} hedefleriyle tek komutta çalıştırır.

\subsection{Ön Koşullar}

Docker Desktop + Kubernetes, \texttt{kubectl}, \texttt{helm}, \texttt{python3}, \texttt{pip}, \texttt{make} ve \texttt{docker} gereklidir. Depo kökünde \texttt{pip install -r requirements-dev.txt} ile Python bağımlılıkları kurulmalıdır.

\subsection{Bootstrap}

Kubernetes kümesinde Metrics Server kurulumu gerçekleştirilir, PHPA operatörü Helm ile dağıtılır, örnek uygulama ve yük üreticisi devreye alınır. Bu adım, tüm deneylerin standart bir ortamda yürütülmesini sağlar.

\noindent \textit{Komut:} \texttt{make -C labs bootstrap}

\subsection{Çevrimdışı Kıyaslama}

Sentetik veri üretimi ile 48 saat eğitim ve 12 saat test verisi oluşturulur. Python eğitim betikleri GBDT, XGBoost, CatBoost ve VAR için çalıştırılır; RMSE/MAE/MAPE metrikleri toplanır ve \texttt{output/offline} altında özetlenir. Böylece modellerin saf kestirim kabiliyetleri kontrol altında karşılaştırılır.

\noindent \textit{Komut:} \texttt{make -C labs offline}

\subsection{Çevrimiçi Kıyaslama}

Gerçek Kubernetes kümesi üzerinde, standart HPA ve her PHPA modeli için 240 saniyelik senaryolar sıra ile çalıştırılır. Replika sayısı, CPU kullanımı ve ölçeklenme olayları zaman damgaları ile kaydedilir. Deney sonunda \texttt{analyze\_online.py} özet CSV ve Markdown çıktıları üretir.

\noindent \textit{Komut:} \texttt{make -C labs online}

\subsection{Raporlama ve Temizlik}

\texttt{generate\_report.py}, çevrimdışı ve çevrimiçi çıktıları birleştirerek \texttt{output/report.md} dosyasını üretir. Deneyler tamamlandıktan sonra \texttt{make destroy} ile operatör, CRD ve örnek kaynaklar temizlenir.

\noindent \textit{Komutlar:} \texttt{make -C labs report} ve \texttt{make -C labs destroy}

\section{Bulgular — Çevrimdışı (Offline Results)}

Sentetik veri üzerinde yapılan deneylerde, aşağıdaki hata metrikleri elde edilmiştir. VAR'ın tek değişkenli sentetik veride kararsızlığa düştüğü not edilmelidir; bu model çok değişkenli girişler için tasarlanmıştır.

\begin{table}[h]
    \centering
    \caption{Offline deneylerde raporlanan hata ölçümleri (RMSE, MAE, MAPE).}
    \label{tab:offline}
    \begin{tabular}{@{}lccc@{}}
        \toprule
        Model & RMSE & MAE & MAPE (\%) \\
        \midrule
        GBDT & 1.98 & 1.56 & 6.13 \\
        XGBoost & 2.31 & 1.85 & 7.42 \\
        CatBoost & 5.55 & 4.56 & 21.36 \\
        VAR & $5.75\times10^{18}$ & $3.59\times10^{18}$ & $1.35\times10^{19}$ \\
        \bottomrule
    \end{tabular}
\end{table}

GBDT ve XGBoost, düşük hata oranları ve hızlı eğitim süreleriyle pratik kullanım için öne çıkmıştır. CatBoost'un performansı, sentetik tek değişkenli veride sınırlı kalmış; gerçek dünyada kategorik/türetik özelliklerle daha iyi sonuçlar beklenmektedir. VAR ise çok değişkenli bağlam gereksinimi nedeniyle tek değişkenli senaryoda uygun değildir.


\section{Bulgular — Çevrimiçi (Online Results)}

Gerçek Kubernetes kümesi üzerinde yürütülen 240 saniyelik senaryolarda, standart HPA ve altı farklı PHPA modeli karşılaştırılmıştır.

\begin{table}[h]
    \centering
    \caption{240 saniyelik çevrimiçi senaryolarda ölçülen göstergeler (Online benchmark metrics).}
    \label{tab:online}
    \begin{tabular}{@{}lcccccc@{}}
        \toprule
        Senaryo & Süre (s) & Zirve Replika & İlk Ölçekleme (s) & CPU Ort. (m) & CPU Zirve (m) & Ort. Replika \\
        \midrule
        Standart HPA & 236 & 5.0 & 41.0 & 342.6 & 536.0 & 2.64 \\
        PHPA Linear & 236 & 10.0 & 10.2 & 525.0 & 635.0 & 8.72 \\
        PHPA Holt-Winters & 236 & 8.0 & 71.9 & 405.4 & 636.0 & 4.77 \\
        PHPA GBDT & 236 & 5.0 & 20.5 & 460.3 & 558.0 & 4.15 \\
        PHPA CatBoost & 236 & 5.0 & 30.8 & 486.4 & 559.0 & 4.36 \\
        PHPA VAR & 236 & 5.0 & 41.1 & 417.3 & 577.0 & 3.81 \\
        PHPA XGBoost & 236 & 5.0 & 51.3 & 496.0 & 594.0 & 4.57 \\
        \bottomrule
    \end{tabular}
\end{table}

Tüm PHPA modelleri, standart HPA'ya kıyasla ilk ölçeklendirme zamanını öne çekmiştir. Linear model en hızlı tepkiyi vermiş; ancak yüksek ortalama replika ve CPU değeri ile maliyeti artırmıştır. GBDT ve XGBoost dengeli bir maliyet-tepki süresi bileşimi sunarken, Holt–Winters mevsimsellik içeren yüklerde uygun bir adaydır. VAR, muhafazakâr ortalama replika davranışıyla kaynak tasarrufu sağlayabilir.


\section{Tartışma (Discussion)}

Elde edilen bulgular, tahmine dayalı ölçeklendirmenin reaktif HPA yaklaşımının sınırlamalarını pratikte aşabildiğini göstermektedir. Özellikle ilk ölçeklendirme kararının 10–51 saniye aralığına çekilmesi, kısa süreli piklerde gecikme kaynaklı performans kayıplarını azaltır. Ancak bu kazanımın maliyet tarafına yansıdığı (daha yüksek ortalama replika/CPU) unutulmamalıdır. Bu nedenle model seçimi, uygulamanın iş hedefleri (gecikme toleransı, maliyet bütçesi, SLA) ile uyumlu olmalıdır.

\subsection{Çevrimdışı Bulguların Yorumu}

GBDT ve XGBoost, düşük hata ve hızlı eğitim/çıkarım özellikleri ile üretim dostu seçeneklerdir. CatBoost, kategorik/zengin özellik seti gereksinimi nedeniyle tek değişkenli sentetik senaryoda geri kalsa da gerçek veri bağlamında güçlenebilir. VAR'ın tek değişkenli senaryoda başarısız olması beklenen bir sonuçtur; modelin doğası gereği çok değişkenli bağlam (CPU, bellek, ağ trafiği, I/O) ile beslendiğinde anlamlı hale gelir.

\subsection{Çevrimiçi Bulguların Yorumu}

Linear model, kritik servislerde gerekli olabilecek agresif ön-ölçeklemeyi sağlar; ancak maliyetli olabilir. GBDT/XGBoost, çoğu mikro servis için dengeli profildir. Holt–Winters, mevsimsellik içeren iş yüklerinde güçlüdür. VAR, kaynak kısıtı baskın senaryolarda (maliyet odaklı) düşünülebilir; daha fazla bağlamsal metrik ile güçlendirilmelidir.

\subsection{(Bağlam) LLM Entegrasyonunun Katkısı}

LLM tabanlı örüntü tanıma, önceki çalışmanın bir parçası olarak operasyon ekibinin manuel analiz yükünü azaltır ve doğru model sınıfının hızlı seçimini kolaylaştırır. %96,7 doğruluk seviyesi, pratikte yüksek bir yardımcı sinyal anlamına gelmektedir. Not: Bu makalede LLM uygulanmamış, yalnızca bağlamsal arka plan olarak ele alınmıştır.

\subsection{Pratik Öneriler}

\begin{itemize}[noitemsep]
  \item \textbf{Model Seçimi:} Hız-kalite-maliyet üçlemesini göz önünde bulundurun; kritik uçlar için Linear/GBDT, genel kullanım için XGBoost/CatBoost, mevsimsellik için Holt–Winters.
  \item \textbf{Çok Değişkenlilik:} VAR ve benzeri yöntemler için CPU, bellek, ağ, disk I/O gibi ek metrikleri veri akışına dahil edin.
  \item \textbf{Hiperparametreler:} Temporal CV ve erken durdurma ile periyodik yeniden ayarlama planlayın.
  \item \textbf{Operasyon:} `decisionType` (max/mean/median) stratejisini servisin SLA’ına göre seçin; agresif veya muhafazakâr politika belirleyin.
  \item \textbf{Gözlemleme:} Model performansını sürekli izleyin; dağıtım sonrası drift/evrim için uyarı eşiği ve yeniden eğitim tetikleyicileri belirleyin.
\end{itemize}

\section{Sınırlılıklar ve Tehditler (Limitations and Threats)}

\textbf{Sentetik Veri:} Çevrimdışı çalışmalarda sentetik tek değişkenli veri kullanılmıştır. Gerçek dünyada örüntüler daha karmaşık olabilir; sonuçların üretim verisiyle doğrulanması gereklidir.

\textbf{Tek Değişkenli Senaryolar:} VAR ve benzeri çok değişkenli yöntemlerin potansiyeli tek değişkenli veriyle tam ortaya konamamıştır. Gelecek çalışmalarda çok değişkenli veri akışları planlanmalıdır.

\textbf{Kısa Online Süre:} 240 saniyelik çevrimiçi test penceresi, uzun dönem davranışları temsil etmeyebilir. Daha uzun süreli (gün/hafta) deneyler önerilir.

\textbf{Maliyet Analizi:} Kaynak kullanımı ölçülmüş olsa da kapsamlı bir bulut maliyet analizi dış kapsamda kalmıştır. Farklı sağlayıcı ve fiyatlandırma modelleriyle birlikte değerlendirme yapılmalıdır.


\section{Simgeler (Symbols)}

Bu bölümde, makalede kullanılan matematiksel simgeler ve kısaltmalar alfabetik sıraya göre açıklanmaktadır.

\subsection{Latin Harfleri (Latin Letters)}

\begin{tabular}{ll}
$A_k$ & Sinüzoidal bileşenin genliği (Amplitude of sinusoidal component) \\
$B$ & Temel yük seviyesi (Base load level) \\
$B_i$ & Patlama yoğunluğu (Burst intensity) \\
$d_i$ & Patlama süresi (Burst duration) \\
$f(t)$ & Büyüme fonksiyonu (Growth function) \\
$g(t)$ & Patlama şekil fonksiyonu (Burst shape function) \\
$G$ & Büyüme oranı (Growth rate) \\
$h_t$ & Günün saati (Hour of day) \\
$K$ & Sinüzoidal bileşen sayısı (Number of sinusoidal components) \\
$L_t$ & Mevcut seviye indeksi (Current level index) \\
MAE & Ortalama mutlak hata (Mean Absolute Error) \\
MAPE & Ortalama mutlak yüzde hatası (Mean Absolute Percentage Error) \\
$N_b$ & Patlama sayısı (Number of bursts) \\
$N_t$ & Gaussian gürültü (Gaussian noise) \\
$P_t$ & Zamanda $t$ anındaki pod sayısı (Pod count at time $t$) \\
$P_{high}$ & Yüksek yük seviyesi (High load level) \\
$P_{low}$ & Düşük yük seviyesi (Low load level) \\
RMSE & Kök ortalama kare hatası (Root Mean Square Error) \\
$S$ & Günlük dalgalanma genliği (Daily fluctuation amplitude) \\
$S_t$ & Durum değişkeni (State variable) \\
$S_{step}$ & Seviye adım büyüklüğü (Level step size) \\
$t$ & Zaman (Time) \\
$t_i$ & Patlama başlangıç zamanı (Burst start time) \\
$T_k$ & Periyot (Period) \\
\end{tabular}

\subsection{Yunan Harfleri (Greek Letters)}

\begin{tabular}{ll}
$\phi_k$ & Faz kayması (Phase shift) \\
$\theta_i$ & Model hiperparametreleri (Model hyperparameters) \\
\end{tabular}

\subsection{Kısaltmalar (Abbreviations)}

\begin{tabular}{ll}
API & Application Programming Interface \\
BIC & Bayesian Information Criterion \\
CatBoost & Categorical Boosting \\
CI/CD & Continuous Integration / Continuous Deployment \\
CPU & Central Processing Unit \\
CRD & Custom Resource Definition \\
CSV & Comma-Separated Values \\
GBDT & Gradient Boosted Decision Trees \\
HPA & Horizontal Pod Autoscaler \\
LLM & Large Language Model (Büyük Dil Modeli) \\
PHPA & Predictive Horizontal Pod Autoscaler \\
VAR & Vector Autoregression \\
XGBoost & Extreme Gradient Boosting \\
YAML & YAML Ain't Markup Language \\
\end{tabular}


\section{Sonuçlar ve Gelecek Çalışmalar (Conclusions and Future Work)}

Bu çalışmada, örüntü-farkındalıklı tahmine dayalı yatay pod otomatik ölçeklendirici (PHPA) çerçevesi syswe ad alanına taşınmış, üretim kullanımına uygun olacak şekilde Kubernetes operatörü ile bütünleştirilmiş ve çevrimdışı/çevrimiçi deneylerle doğrulanmıştır. Altı temel yük örüntüsü için matematiksel formülasyonlar ve geniş kapsamlı veri seti, örüntü-spesifik model seçimi ilkesiyle birleştirilmiştir. LLM tabanlı otomatik örüntü tanıma önceki çalışmanın bir parçasıdır ve bu makalenin kapsamı dışındadır. Operatörde GBDT, XGBoost, CatBoost ve VAR modelleri aktif edilmiştir.

Deneysel bulgular, GBDT ve XGBoost'un pratikte dengeli ve güçlü seçenekler olduğunu; Linear'ın hızlı fakat maliyetli, Holt–Winters'ın mevsimsellik için uygun, VAR'ın çok değişkenli bağlamla güçlenen bir aday olduğunu göstermektedir. Tüm PHPA modelleri, standart HPA'ya kıyasla ölçeklendirme kararlarını öne çekerek gecikme kaynaklı performans kayıplarını azaltmıştır.

\subsection{Gelecek Çalışmalar}

\begin{itemize}[noitemsep]
  \item \textbf{Gerçek Dünya Doğrulaması:} Farklı üretim ortamlarından toplanan verilerle uzun süreli (gün/hafta) deneyler.
  \item \textbf{Gelişmiş LLM Entegrasyonu:} İleri prompt mühendisliği, ansambllar ve birden çok LLM ile karşılaştırmalı değerlendirme.
  \item \textbf{Örüntü Evrimi:} Örüntü değişimlerinin otomatik tespiti ve dinamik model geçişleri.
  \item \textbf{Maliyet Optimizasyonu:} Bulut sağlayıcılarına özgü fiyatlandırmalarla çok amaçlı maliyet-performans-doğruluk optimizasyonu.
  \item \textbf{Çok Bulut ve Kenar Bilişim:} Dağıtık kaynak yönetimi ve hiyerarşik ölçeklendirme stratejileri.
  \item \textbf{İş Hedefi Entegrasyonu:} SLA hedefleri ve ekonomik kısıtlarla birlikte karar politikalarının uyarlanması.
\end{itemize}

\section*{Teşekkür (Acknowledgement)}

Bu çalışma, TÜBİTAK 1005 tarafından desteklenmiştir. Kocaeli Üniversitesi Bilişim Sistemleri Mühendisliği Bölümü'ne sağladığı altyapı ve destek için teşekkür ederiz. Kullanılan açık kaynak projelerin (Kubernetes, XGBoost, CatBoost, scikit-learn, statsmodels, Prophet) topluluklarına katkıları için teşekkür ederiz.

\section*{Kaynaklar (References)}

\begin{enumerate}[label={[\arabic*]}]

\item Burns B., Beda J., Hightower K., Kubernetes: Up and Running, O'Reilly Media, 2019.
\item Kubernetes Authors, Kubernetes Documentation: Horizontal Pod Autoscaling, https://kubernetes.io/docs, Erişim: Ocak 15, 2025.
\item Lorido-Botran T., Miguel-Alonso J., Lozano J.A., A Review of Auto-scaling Techniques for Elastic Applications in Cloud Environments, J. Grid Comput., 12 (4), 559-592, 2014.
\item Qu C., Calheiros R.N., Buyya R., Auto-scaling Web Applications in Clouds: A Taxonomy and Survey, ACM Comput. Surv., 51 (4), 1-33, 2018.
\item Box G.E.P., Jenkins G.M., Reinsel G.C., Ljung G.M., Time Series Analysis: Forecasting and Control, Wiley, 2015.
\item Hyndman R.J., Athanasopoulos G., Forecasting: Principles and Practice, OTexts, 2021.
\item Jiang J., Lu J., Zhang G., Long G., Optimal Cloud Resource Auto-Scaling for Web Applications, CCGrid, 2013.
\item Roy N., Dubey A., Gokhale A., Efficient Autoscaling in the Cloud Using Predictive Models, IEEE Cloud, 2011.
\item Duggan M., Mason K., Duggan J., Howley E., Barrett E., Predicting Host CPU Utilization in Cloud Computing Using RNNs, FNC, 2017.
\item Dang-Quang N.M., Yoo M., Bi-LSTM-based Autoscaling for Kubernetes, Appl. Sci., 11 (9), 3835, 2021.
\item Imdoukh M., Ahmad I., Alfailakawi M.G., ML-Based Auto-Scaling for Containerized Applications, Neural Comput. Appl., 2020.
\item Jin M. ve diğ., Time-LLM: Time Series Forecasting by Reprogramming LLMs, ICLR, 2024.
\item Gruver N., Finzi M., Qiu S., Wilson A.G., LLMs Are Zero-Shot Time Series Forecasters, NeurIPS, 2023.
\item Zhou T. ve diğ., One Fits All: Power General Time Series Analysis by Pretrained LM, NeurIPS, 2023.
\item Duman C., Kubernetes Üzerinde Yük Örüntüsü Farkındalıklı Tahmine Dayalı Otomatik Ölçeklendirme Çerçevesinin Büyük Dil Modeli Entegrasyonu ile Geliştirilmesi, Yüksek Lisans Tezi, KOU, 2025.
\item Herbst N.R., Kounev S., Reussner R., Elasticity in Cloud Computing: What It Is, ICAC, 2013.
\item Mao M., Humphrey M., Auto-scaling to Minimize Cost and Meet Deadlines, SC, 2011.
\item Chen T., Guestrin C., XGBoost: A Scalable Tree Boosting System, KDD, 2016.
\item Prokhorenkova L. ve diğ., CatBoost: Unbiased Boosting with Categorical Features, NeurIPS, 2018.
\item Taylor S.J., Letham B., Forecasting at Scale, American Statistician, 2018.

\end{enumerate}



\end{document}
