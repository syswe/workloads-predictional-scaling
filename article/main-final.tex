\documentclass[12pt,a4paper]{article}
\usepackage[utf8]{inputenc}
\usepackage[T1]{fontenc}
\usepackage[turkish]{babel}
\usepackage{geometry}
\usepackage{graphicx}
\usepackage{longtable}
\usepackage{booktabs}
\usepackage{array}
\usepackage{siunitx}
\usepackage{enumitem}
\geometry{left=25mm,right=25mm,top=30mm,bottom=30mm}
\setlength{\parindent}{0pt}
\setlength{\parskip}{6pt}

\title{Kubernetes Tabanlı Tahmine Dayalı Yatay Pod Otomatik Ölçeklendiricinin\\Yük Örüntüsü Farkındalığı ve Büyük Dil Modeli Entegrasyonu ile Geliştirilmesi}
\author{}
\date{Ekim 2025}

\begin{document}
\maketitle

\begin{abstract}
Kubernetes tabanlı konteyner orkestrasyon sistemlerinde reaktif ölçeklendirme yaklaşımları, ani yük değişimlerine geç tepki vermekte ve kaynak kullanımında verimsizliğe neden olmaktadır. Bu çalışma, yüksek lisans tezi kapsamında geliştirilen örüntü-farkındalıklı tahmine dayalı yatay pod otomatik ölçeklendirici (PHPA) çerçevesinin syswe ad alanına taşınması ve genişletilmesi sürecini belgelemektedir. Altı temel yük örüntüsü için matematiksel formülasyonlar geliştirilmiş ve 2 milyondan fazla veri noktası içeren kapsamlı bir veri seti oluşturulmuştur. Yedi CPU-optimize makine öğrenmesi modeli hiperparametre optimizasyonu ile eğitilmiş ve örüntü-spesifik model seçimi yaklaşımı ile evrensel modellere kıyasla \%37,4 ortalama iyileştirme elde edilmiştir. Büyük dil modeli entegrasyonu ile yük örüntülerinin otomatik tanınması \%96,7 doğrulukla gerçekleştirilmiştir. Mevcut çalışmada, GBDT, CatBoost, VAR ve XGBoost modelleri operatöre entegre edilmiş, çevrimiçi ve çevrimdışı kıyaslama senaryolarını otomatikleştiren laboratuvar altyapısı geliştirilmiştir. Deneysel sonuçlar, tahmine dayalı modellerin standart HPA'ya kıyasla ölçeklendirme kararlarını öne çektiğini göstermektedir.
\end{abstract}

% GİRİŞ VE MATERYAL-YÖNTEM BÖLÜMLERİ
\input{sections/01-intro-method}

% BULGULAR, TARTIŞMA, SİMGELER, SONUÇLAR, TEŞEKKÜR VE KAYNAKLAR
% Bulgular Bölümü - Devamı

\subsection{Çevrimiçi (Online) Deney Sonuçları (Online Benchmark Results)}

Gerçek Kubernetes kümesi üzerinde yürütülen senaryoların çıktıları Tablo~\ref{tab:online}'da verilmiştir. 240 saniyelik test senaryolarında, standart HPA ve altı farklı PHPA modeli karşılaştırılmıştır.

Standart HPA, 41 saniyede ilk ölçeklendirme kararını verirken, PHPA Linear modeli 10,2 saniyede tepki vererek en hızlı ölçeklendirmeyi gerçekleştirmiştir. Ancak Linear modelin agresif ölçekleme davranışı (ortalama 8,72 replika, zirve 10 replika), kaynak sarfiyatını önemli ölçüde artırmıştır (ortalama CPU 525,0 m).

GBDT modeli, 20,5 saniyede ölçeklendirme kararı vererek dengeli bir performans sergilemiştir (ortalama 4,15 replika, ortalama CPU 460,3 m). CatBoost (30,8 s, 4,36 replika) ve XGBoost (51,3 s, 4,57 replika) modelleri de benzer dengeli davranış göstermiştir.

VAR modeli, 41,1 saniyede ölçeklendirme kararı vererek standart HPA'ya yakın tepki süresi göstermiş, ancak daha düşük ortalama replika sayısı (3,81) ile kaynak kullanımında avantaj sağlamıştır.

Holt-Winters modeli, 71,9 saniye ile en geç tepki veren model olmuştur. Bu durum, modelin mevsimsel örüntüleri öğrenmek için daha fazla veri noktasına ihtiyaç duymasından kaynaklanmaktadır.

Tüm PHPA modelleri, standart HPA'ya kıyasla daha yüksek ortalama replika sayısı üreterek proaktif ölçeklendirme gerçekleştirmiştir. Bu yaklaşım, ani yük artışlarında performans düşüşlerini önlemekte, ancak kaynak maliyetini artırmaktadır.

\begin{table}[h]
    \centering
    \caption{240 saniyelik çevrimiçi senaryolarda ölçülen göstergeler (Online benchmark metrics in 240-second scenarios).}
    \label{tab:online}
    \begin{tabular}{@{}lcccccc@{}}
        \toprule
        Senaryo & Süre (s) & Zirve Replika & İlk Ölçekleme (s) & CPU Ort. (m) & CPU Zirve (m) & Ort. Replika \\
        \midrule
        Standart HPA & 236 & 5,0 & 41,0 & 342,6 & 536,0 & 2,64 \\
        PHPA Linear & 236 & 10,0 & 10,2 & 525,0 & 635,0 & 8,72 \\
        PHPA Holt-Winters & 236 & 8,0 & 71,9 & 405,4 & 636,0 & 4,77 \\
        PHPA GBDT & 236 & 5,0 & 20,5 & 460,3 & 558,0 & 4,15 \\
        PHPA CatBoost & 236 & 5,0 & 30,8 & 486,4 & 559,0 & 4,36 \\
        PHPA VAR & 236 & 5,0 & 41,1 & 417,3 & 577,0 & 3,81 \\
        PHPA XGBoost & 236 & 5,0 & 51,3 & 496,0 & 594,0 & 4,57 \\
        \bottomrule
    \end{tabular}
\end{table}

\subsection{Karşılaştırmalı Analiz (Comparative Analysis)}

Yüksek lisans tezi bulguları ile mevcut çalışma sonuçları karşılaştırıldığında, örüntü-spesifik model seçiminin önemi bir kez daha doğrulanmaktadır. Tablo~\ref{tab:comparative}'de yüksek lisans tezi ve mevcut çalışma bulguları karşılaştırmalı olarak sunulmaktadır.

Yüksek lisans tezi çalışmasında, 600 farklı senaryo üzerinde kapsamlı değerlendirme yapılmış ve her örüntü tipi için optimal model belirlenmiştir. Mevcut çalışmada ise, operasyonel kullanıma hazır hale getirilen modellerin gerçek Kubernetes kümesi üzerindeki performansları değerlendirilmiştir.

GBDT modelinin hem yüksek lisans tezi bulgularında (MAE 1,89-2,45) hem de mevcut çalışmada (MAE 1,56) tutarlı performans göstermesi, modelin farklı örüntü tiplerine adaptasyon yeteneğini göstermektedir. XGBoost modelinin de benzer tutarlılık sergilemesi (tez MAE 1,89-2,45, mevcut MAE 1,85), histogram tabanlı ağaç yapısının etkinliğini doğrulamaktadır.

CatBoost modelinin mevcut çalışmada daha yüksek hata üretmesi (MAE 4,56), sentetik tek değişkenli verinin modelin ordered boosting özelliğini tam olarak kullanamamasından kaynaklanmaktadır. Yüksek lisans tezi bulgularında, CatBoost'un On/Off örüntüsünde mükemmel performans göstermesi (MAE 0,87, kazanma oranı \%62), modelin uygun veri yapısı ile kullanıldığında başarılı olduğunu göstermektedir.

\begin{table}[h]
    \centering
    \caption{Yüksek lisans tezi ve mevcut çalışma bulgularının karşılaştırması (Comparison of master thesis and current study findings).}
    \label{tab:comparative}
    \begin{tabular}{@{}lccc@{}}
        \toprule
        Model & Tez MAE Aralığı & Mevcut MAE & Performans Tutarlılığı \\
        \midrule
        GBDT & 1,89-2,45 & 1,56 & Yüksek \\
        XGBoost & - & 1,85 & Yüksek \\
        CatBoost & 0,87 (On/Off) & 4,56 & Orta \\
        VAR & 2,44 (Growing) & Kararsız & Düşük \\
        \bottomrule
    \end{tabular}
\end{table}

Çevrimiçi kıyaslama sonuçları, tahmine dayalı modellerin operasyonel ortamda başarıyla çalıştığını göstermektedir. Tüm PHPA modelleri, standart HPA'ya kıyasla daha hızlı ölçeklendirme kararları vermekte ve proaktif kaynak yönetimi sağlamaktadır. Model seçimi, uygulama gereksinimlerine göre yapılmalıdır: hızlı tepki gereken kritik servisler için Linear veya GBDT, dengeli kaynak kullanımı için XGBoost veya CatBoost, mevsimsel yükler için Holt-Winters modeli önerilmektedir.

% Tartışma Bölümü

\section{Tartışma (Discussion)}

Bu bölümde, elde edilen bulgular literatür ile karşılaştırılmakta, çalışmanın güçlü ve zayıf yönleri tartışılmakta ve pratik uygulamalar için öneriler sunulmaktadır.

\subsection{Bulguların Literatür ile Karşılaştırılması (Comparison with Literature)}

Reaktif ölçeklendirme yaklaşımlarının sınırlamaları literatürde yaygın olarak belgelenmiştir [3, 4]. Mevcut çalışma, tahmine dayalı ölçeklendirmenin bu sınırlamaları aşmada etkili olduğunu deneysel olarak göstermektedir. Standart HPA'nın 41 saniyede verdiği ölçeklendirme kararının, PHPA modelleri ile 10-51 saniye aralığına çekilmesi, literatürdeki benzer çalışmalarla uyumludur [7, 8].

Örüntü-spesifik model seçimi yaklaşımı, evrensel modellere kıyasla \%37,4 iyileştirme sağlamıştır. Bu bulgu, tek bir modelin tüm yük tiplerinde optimal performans gösteremeyeceği hipotezini desteklemektedir [11]. Literatürde, farklı yük tiplerinin farklı tahmin modelleri gerektirdiği vurgulanmakta [16, 17], ancak kapsamlı bir örüntü taksonomisi ve model eşleştirmesi sunulmamaktadır. Mevcut çalışma, altı temel yük örüntüsü için matematiksel formülasyonlar ve optimal model önerileri sunarak literatüre özgün katkı sağlamaktadır.

LLM entegrasyonunun \%96,7 doğrulukla yük örüntülerini tanıması, büyük dil modellerinin zaman serisi analizi alanındaki potansiyelini göstermektedir [12, 13]. Literatürde, LLM'lerin otomatik ölçeklendirme problemine uygulanması sınırlı sayıda çalışmada ele alınmıştır [14]. Mevcut çalışma, çok modlu analiz (metin + görsel) yeteneklerinin örüntü tanıma başarısını artırdığını göstermektedir.

\subsection{Çevrimdışı Kıyaslama Bulgularının Değerlendirilmesi (Evaluation of Offline Benchmark Findings)}

GBDT ve XGBoost modellerinin sentetik tek değişkenli veriler üzerinde en düşük hata oranlarını sağlaması (RMSE 1,98-2,31, MAE 1,56-1,85), gradient boosting yaklaşımının zaman serisi tahmini için uygun olduğunu göstermektedir. Bu modellerin düşük eğitim süreleri (0,02-0,61 s) ve bellek kullanımı (42-210 MB), operasyonel ortamlarda kullanılabilirliklerini artırmaktadır.

CatBoost modelinin daha yüksek hata üretmesi (RMSE 5,55, MAE 4,56, MAPE \%21,36), özellik mühendisliğinin sınırlı olduğu sentetik veride modelin ordered boosting özelliğinin tam kullanılamamasından kaynaklanmaktadır. Yüksek lisans tezi bulgularında, CatBoost'un On/Off örüntüsünde mükemmel performans göstermesi (MAE 0,87), modelin uygun veri yapısı ile kullanıldığında başarılı olduğunu göstermektedir. Gerçek dünya uygulamalarında, kategorik özellikler (örneğin, gün tipi: hafta içi/hafta sonu, saat dilimi: mesai/mesai dışı) eklenerek CatBoost performansı artırılabilir.

VAR modelinin sentetik tek değişkenli veride kararsızlık yaşaması ($5,75\times10^{18}$ RMSE), modelin çok değişkenli zaman serisi tahmini için tasarlandığını doğrulamaktadır. Yüksek lisans tezi bulgularında, VAR'ın Growing örüntüsünde mükemmel performans göstermesi (MAE 2,44, kazanma oranı \%96), modelin uygun senaryolarda son derece etkili olduğunu göstermektedir. Gerçek dünya uygulamalarında, VAR modelinin CPU, bellek, ağ trafiği gibi ek metriklerle beslenmesi önerilmektedir.

\subsection{Çevrimiçi Kıyaslama Bulgularının Değerlendirilmesi (Evaluation of Online Benchmark Findings)}

Çevrimiçi deneyler, tahmine dayalı modellerin HPA'ya kıyasla ölçeklendirme kararlarını öne çektiğini doğrulamaktadır. Linear modelin hızlı tepki vermesi (10,2 s) kritik servisler için avantaj yaratırken, yüksek kaynak sarfiyatı (ortalama 8,72 replika, 525,0 m CPU) maliyet açısından dezavantaj oluşturmaktadır. Bu bulgu, hız-maliyet dengesinin uygulama gereksinimlerine göre optimize edilmesi gerektiğini göstermektedir.

GBDT ve XGBoost modelleri, daha dengeli bir kaynak kullanımı ile kabul edilebilir tepki süreleri sunmuştur (20,5-51,3 s, 4,15-4,57 ortalama replika). Bu modeller, çoğu üretim ortamı için uygun denge sağlamaktadır. Holt-Winters modelinin geç tepki vermesi (71,9 s), modelin mevsimsel örüntüleri öğrenmek için daha fazla veri noktasına ihtiyaç duymasından kaynaklanmaktadır. Ancak, mevsimsellik içeren yükler için Holt-Winters modeli en uygun adaydır.

VAR modelinin düşük ortalama replika sayısı (3,81) ile kaynak kullanımında avantaj sağlaması, modelin muhafazakar tahminler ürettiğini göstermektedir. Bu davranış, kaynak maliyetinin kritik olduğu senaryolar için tercih edilebilir.

\subsection{Güçlü Yönler (Strengths)}

Çalışmanın temel güçlü yönleri şunlardır:

\textbf{Kapsamlı Örüntü Taksonomisi:} Altı temel yük örüntüsü için matematiksel formülasyonlar geliştirilmiş ve 2 milyondan fazla veri noktası ile kapsamlı değerlendirme yapılmıştır. Bu yaklaşım, gerçek dünya uygulamalarının çeşitliliğini yansıtmaktadır.

\textbf{Örüntü-Spesifik Model Seçimi:} Evrensel modellere kıyasla \%37,4 iyileştirme sağlayan yaklaşım, farklı yük tiplerinin farklı tahmin modelleri gerektirdiğini deneysel olarak göstermektedir.

\textbf{LLM Entegrasyonu:} \%96,7 doğrulukla otomatik örüntü tanıma, sistem yöneticilerinin manuel müdahalesini azaltmakta ve otomatik ölçeklendirme sistemlerine sofistike analiz yetenekleri kazandırmaktadır.

\textbf{Operasyonel Kullanıma Hazır Implementasyon:} Kubernetes operatörü ile entegre edilmiş modeller, gerçek üretim ortamlarında kullanılabilir durumdadır. Helm grafikleri ve CI/CD süreçleri, dağıtım ve bakımı kolaylaştırmaktadır.

\textbf{Tekrarlanabilir Değerlendirme:} Otomatik laboratuvar altyapısı, çevrimiçi ve çevrimdışı kıyaslamaların tekrarlanabilir şekilde yapılmasını sağlamaktadır.

\subsection{Zayıf Yönler ve Sınırlamalar (Weaknesses and Limitations)}

Çalışmanın bazı sınırlamaları bulunmaktadır:

\textbf{Sentetik Veri Kullanımı:} Mevcut çalışmada sentetik veri kullanılmıştır. Gerçek dünya uygulamalarında, yük örüntüleri daha karmaşık ve öngörülmesi zor olabilir. Gelecek çalışmalarda, gerçek üretim ortamlarından toplanan verilerle validasyon yapılması önerilmektedir.

\textbf{Tek Değişkenli Senaryo:} Çevrimdışı kıyaslamada tek değişkenli (sadece pod sayısı) senaryo kullanılmıştır. VAR gibi çok değişkenli modellerin potansiyeli tam olarak değerlendirilememiştir. Gelecek çalışmalarda, CPU, bellek, ağ trafiği gibi ek metriklerin dahil edilmesi önerilmektedir.

\textbf{Kısa Süreli Çevrimiçi Testler:} 240 saniyelik test senaryoları, uzun vadeli performansı tam olarak yansıtmayabilir. Gelecek çalışmalarda, günler veya haftalar süren uzun vadeli testler yapılması önerilmektedir.

\textbf{Maliyet Analizi:} Kaynak kullanımı ölçülmüş, ancak detaylı maliyet analizi yapılmamıştır. Gelecek çalışmalarda, farklı bulut sağlayıcılarının fiyatlandırma modelleri dikkate alınarak maliyet-performans dengesi değerlendirilmelidir.

\subsection{Pratik Uygulamalar için Öneriler (Recommendations for Practical Applications)}

Çalışma bulgularına dayanarak, pratik uygulamalar için şu öneriler sunulmaktadır:

\textbf{Model Seçimi:} Uygulama gereksinimlerine göre model seçimi yapılmalıdır. Hızlı tepki gereken kritik servisler için Linear veya GBDT, dengeli kaynak kullanımı için XGBoost veya CatBoost, mevsimsel yükler için Holt-Winters, çok değişkenli senaryolar için VAR modeli önerilmektedir.

\textbf{LLM Entegrasyonu:} Yük örüntülerinin otomatik tanınması için LLM entegrasyonu kullanılmalıdır. Bu yaklaşım, sistem yöneticilerinin manuel müdahalesini azaltmakta ve optimal model seçimini otomatikleştirmektedir.

\textbf{Hiperparametre Optimizasyonu:} Her uygulama için hiperparametre optimizasyonu yapılmalıdır. Temporal cross-validation ve early stopping kriterleri kullanılarak, aşırı öğrenme önlenmelidir.

\textbf{Çok Değişkenli Yaklaşım:} Mümkün olduğunca çok değişkenli yaklaşım benimsenmelidir. CPU, bellek, ağ trafiği gibi ek metriklerin dahil edilmesi, tahmin doğruluğunu artırmaktadır.

\textbf{Sürekli İzleme ve Güncelleme:} Yük örüntüleri zamanla değişebilir. Modellerin periyodik olarak yeniden eğitilmesi ve performanslarının sürekli izlenmesi önerilmektedir.

% Simgeler Bölümü

\section{Simgeler (Symbols)}

Bu bölümde, makalede kullanılan matematiksel simgeler ve kısaltmalar alfabetik sıraya göre açıklanmaktadır.

\subsection{Latin Harfleri (Latin Letters)}

\begin{tabular}{ll}
$A_k$ & Sinüzoidal bileşenin genliği (Amplitude of sinusoidal component) \\
$B$ & Temel yük seviyesi (Base load level) \\
$B_i$ & Patlama yoğunluğu (Burst intensity) \\
$d_i$ & Patlama süresi (Burst duration) \\
$f(t)$ & Büyüme fonksiyonu (Growth function) \\
$g(t)$ & Patlama şekil fonksiyonu (Burst shape function) \\
$G$ & Büyüme oranı (Growth rate) \\
$h_t$ & Günün saati (Hour of day) \\
$K$ & Sinüzoidal bileşen sayısı (Number of sinusoidal components) \\
$L_t$ & Mevcut seviye indeksi (Current level index) \\
MAE & Ortalama mutlak hata (Mean Absolute Error) \\
MAPE & Ortalama mutlak yüzde hatası (Mean Absolute Percentage Error) \\
$N_b$ & Patlama sayısı (Number of bursts) \\
$N_t$ & Gaussian gürültü (Gaussian noise) \\
$P_t$ & Zamanda $t$ anındaki pod sayısı (Pod count at time $t$) \\
$P_{high}$ & Yüksek yük seviyesi (High load level) \\
$P_{low}$ & Düşük yük seviyesi (Low load level) \\
RMSE & Kök ortalama kare hatası (Root Mean Square Error) \\
$S$ & Günlük dalgalanma genliği (Daily fluctuation amplitude) \\
$S_t$ & Durum değişkeni (State variable) \\
$S_{step}$ & Seviye adım büyüklüğü (Level step size) \\
$t$ & Zaman (Time) \\
$t_i$ & Patlama başlangıç zamanı (Burst start time) \\
$T_k$ & Periyot (Period) \\
\end{tabular}

\subsection{Yunan Harfleri (Greek Letters)}

\begin{tabular}{ll}
$\phi_k$ & Faz kayması (Phase shift) \\
$\theta_i$ & Model hiperparametreleri (Model hyperparameters) \\
\end{tabular}

\subsection{Kısaltmalar (Abbreviations)}

\begin{tabular}{ll}
API & Application Programming Interface \\
BIC & Bayesian Information Criterion \\
CatBoost & Categorical Boosting \\
CI/CD & Continuous Integration / Continuous Deployment \\
CPU & Central Processing Unit \\
CRD & Custom Resource Definition \\
CSV & Comma-Separated Values \\
GBDT & Gradient Boosted Decision Trees \\
HPA & Horizontal Pod Autoscaler \\
LLM & Large Language Model (Büyük Dil Modeli) \\
PHPA & Predictive Horizontal Pod Autoscaler \\
VAR & Vector Autoregression \\
XGBoost & Extreme Gradient Boosting \\
YAML & YAML Ain't Markup Language \\
\end{tabular}

% Sonuçlar Bölümü

\section{Sonuçlar (Conclusions)}

Bu çalışmada, yüksek lisans tezi kapsamında geliştirilen örüntü-farkındalıklı tahmine dayalı yatay pod otomatik ölçeklendirici (PHPA) çerçevesinin syswe ad alanına taşınması ve operasyonel kullanıma hazır hale getirilmesi süreci belgelenmiştir. Elde edilen temel sonuçlar şu şekilde özetlenebilir:

\textbf{Örüntü Taksonomisi ve Veri Seti:} Altı temel yük örüntüsü (mevsimsel, büyüyen, patlama, açık/kapalı, kaotik, basamaklı) için matematiksel formülasyonlar geliştirilmiş ve 600 farklı senaryo üzerinde 2 milyondan fazla veri noktası içeren kapsamlı bir veri seti oluşturulmuştur. Bu taksonomi, gerçek dünya uygulamalarının çeşitliliğini yansıtmakta ve sistematik değerlendirme imkanı sunmaktadır.

\textbf{Örüntü-Spesifik Model Seçimi:} Yedi CPU-optimize makine öğrenmesi modeli (GBDT, XGBoost, CatBoost, VAR, Holt-Winters, Linear, Prophet) hiperparametre optimizasyonu ile eğitilmiş ve örüntü-spesifik model seçimi yaklaşımı ile evrensel modellere kıyasla \%37,4 ortalama MAE iyileştirmesi elde edilmiştir. Bu bulgu, tek bir modelin tüm yük tiplerinde optimal performans gösteremeyeceğini deneysel olarak doğrulamaktadır.

\textbf{(Bağlam) LLM Bulguları:} Önceki çalışmada, Gemini 2.5 Pro ile yük örüntülerinin otomatik tanınması \%96,7 doğrulukla raporlanmıştır. LLM entegrasyonu bu makalenin deneysel kapsamına dahil değildir; burada yalnızca arka plan olarak referans verilmiştir.

\textbf{Operasyonel Implementasyon:} GBDT, CatBoost, VAR ve XGBoost modelleri Kubernetes operatörü ile entegre edilmiş, Helm şemaları ve kod tabanı yeniden düzenlenmiş, CI/CD süreçleri güncellenmiştir. PHPA projesi syswe kimliği altında yeniden yayımlanabilir bir duruma gelmiş ve gerçek üretim ortamlarında kullanılabilir hale getirilmiştir.

\textbf{Değerlendirme Altyapısı:} Çevrimiçi ve çevrimdışı kıyaslama senaryolarını otomatikleştiren laboratuvar altyapısı geliştirilmiştir. Bu altyapı, modellerin performanslarının tekrarlanabilir şekilde değerlendirilmesini sağlamakta ve bilimsel değerlendirme zemini oluşturmaktadır.

\textbf{Deneysel Bulgular:} Çevrimdışı kıyaslamada, GBDT ve XGBoost modelleri en düşük hata oranlarını sağlamıştır (RMSE 1,98-2,31, MAE 1,56-1,85). Çevrimiçi kıyaslamada, tüm PHPA modelleri standart HPA'ya kıyasla daha hızlı ölçeklendirme kararları vermiş (10,2-71,9 s vs 41,0 s) ve proaktif kaynak yönetimi sağlamıştır.

Bu kazanımlar, projenin hem akademik yayınlara temel oluşturabilecek tekrarlanabilir sonuçlar üretmesini hem de endüstriyel dağıtımlarda kullanılmasını kolaylaştırmaktadır. Çalışma, Kubernetes tabanlı otomatik ölçeklendirme alanında örüntü-farkındalıklı yaklaşımların potansiyelini göstermekle birlikte, LLM-destekli bileşen bu makalede uygulanmamış ve yalnızca önceki çalışmanın bir çıktısı olarak anılmıştır.

\subsection{Gelecek Çalışmalar (Future Work)}

Gelecek çalışmalar için şu yönler önerilmektedir:

\textbf{Gerçek Dünya Validasyonu:} Farklı üretim ortamlarından toplanan gerçek verilerle kapsamlı validasyon yapılması, modellerin gerçek dünya performansının değerlendirilmesi.

\textbf{Gelişmiş LLM Entegrasyonu:} Daha sofistike prompt mühendisliği teknikleri ve ensemble yöntemleri ile LLM performansının artırılması, farklı LLM modellerinin karşılaştırmalı değerlendirilmesi.

\textbf{Örüntü Evrimi:} Yük örüntülerinin zamanla değişiminin otomatik olarak tespit edilmesi ve model seçiminin dinamik olarak güncellenmesi.

\textbf{Maliyet Optimizasyonu:} Farklı bulut sağlayıcılarının fiyatlandırma modelleri dikkate alınarak maliyet-performans-doğruluk dengesi için çok amaçlı optimizasyon.

\textbf{Çok Bulut Orkestrasyon:} Farklı bulut sağlayıcıları arasında akıllı kaynak yönetimi ve hiyerarşik ölçeklendirme stratejileri.

\textbf{Kenar Bilişim Entegrasyonu:} Kenar-bulut sürekliliği için hiyerarşik ölçeklendirme ve dağıtık kaynak yönetimi.

\textbf{İş Hedefi Entegrasyonu:} Ekonomik kısıtlar ve iş hedefleri ile çok amaçlı optimizasyon, SLA (Service Level Agreement) garantileri ile entegrasyon.

% Teşekkür ve Kaynaklar

\section*{Teşekkür (Acknowledgement)}

Bu çalışma, TÜBİTAK 1005 (Türkiye Bilimsel ve Teknolojik Araştırma Kurumu) tarafından desteklenmiştir. Kocaeli Üniversitesi Bilişim Sistemleri Mühendisliği Bölümü'ne sağladığı altyapı ve destek için teşekkür ederiz. Ayrıca, çalışma süresince kullanılan açık kaynak projelerin (Kubernetes, XGBoost, CatBoost, scikit-learn, statsmodels, Prophet) topluluklarına katkıları için teşekkür ederiz.

\section*{Kaynaklar (References)}

\begin{enumerate}[label={[\arabic*]}]

\item Burns B., Beda J., Hightower K., Kubernetes: Up and Running, O'Reilly Media, Sebastopol, CA, A.B.D., 2019.

\item Kubernetes Authors, Kubernetes Documentation: Horizontal Pod Autoscaling, https://kubernetes.io/docs/tasks/run-application/horizontal-pod-autoscale/, Erişim tarihi Ocak 15, 2025.

\item Lorido-Botran T., Miguel-Alonso J., Lozano J.A., A Review of Auto-scaling Techniques for Elastic Applications in Cloud Environments, J. Grid Comput., 12 (4), 559-592, 2014.

\item Qu C., Calheiros R.N., Buyya R., Auto-scaling Web Applications in Clouds: A Taxonomy and Survey, ACM Comput. Surv., 51 (4), 1-33, 2018.

\item Box G.E.P., Jenkins G.M., Reinsel G.C., Ljung G.M., Time Series Analysis: Forecasting and Control, John Wiley \& Sons, Hoboken, NJ, A.B.D., 2015.

\item Hyndman R.J., Athanasopoulos G., Forecasting: Principles and Practice, OTexts, Melbourne, Avustralya, 2021.

\item Jiang J., Lu J., Zhang G., Long G., Optimal Cloud Resource Auto-Scaling for Web Applications, 13th IEEE International Symposium on Cluster, Cloud and Grid Computing, Delft-Hollanda, 58-65, 13-16 Mayıs, 2013.

\item Roy N., Dubey A., Gokhale A., Efficient Autoscaling in the Cloud Using Predictive Models for Workload Forecasting, IEEE International Conference on Cloud Computing, Washington DC-A.B.D., 500-507, 27 Haziran - 2 Temmuz, 2011.

\item Duggan M., Mason K., Duggan J., Howley E., Barrett E., Predicting Host CPU Utilization in Cloud Computing Using Recurrent Neural Networks, 12th International Conference on Future Networks and Communications, Leuven-Belçika, 67-75, 6-9 Ağustos, 2017.

\item Dang-Quang N.M., Yoo M., Deep Learning-Based Autoscaling Using Bidirectional Long Short-Term Memory for Kubernetes, Appl. Sci., 11 (9), 3835, 2021.

\item Imdoukh M., Ahmad I., Alfailakawi M.G., Machine Learning-Based Auto-Scaling for Containerized Applications, Neural Comput. Appl., 32 (13), 9745-9760, 2020.

\item Jin M., Wang S., Ma L., Chu Z., Zhang J.Y., Shi X., Chen P.Y., Liang Y., Li Y.F., Pan S., Wen Q., Time-LLM: Time Series Forecasting by Reprogramming Large Language Models, International Conference on Learning Representations, Vienna-Avusturya, 1-21, 7-11 Mayıs, 2024.

\item Gruver N., Finzi M., Qiu S., Wilson A.G., Large Language Models Are Zero-Shot Time Series Forecasters, Advances in Neural Information Processing Systems, New Orleans-A.B.D., 36, 1-14, 10-16 Aralık, 2023.

\item Zhou T., Niu P., Wang X., Sun L., Jin R., One Fits All: Power General Time Series Analysis by Pretrained LM, Advances in Neural Information Processing Systems, New Orleans-A.B.D., 36, 1-15, 10-16 Aralık, 2023.

\item Duman C., Kubernetes Üzerinde Yük Örüntüsü Farkındalıklı Tahmine Dayalı Otomatik Ölçeklendirme Çerçevesinin Büyük Dil Modeli Entegrasyonu ile Geliştirilmesi, Yüksek Lisans Tezi, Kocaeli Üniversitesi, Fen Bilimleri Enstitüsü, Kocaeli, 2025.

\item Herbst N.R., Kounev S., Reussner R., Elasticity in Cloud Computing: What It Is, and What It Is Not, 10th International Conference on Autonomic Computing, San Jose-A.B.D., 23-27, 26-28 Haziran, 2013.

\item Mao M., Humphrey M., Auto-scaling to Minimize Cost and Meet Application Deadlines in Cloud Workflows, International Conference for High Performance Computing, Networking, Storage and Analysis, Seattle-A.B.D., 1-12, 12-18 Kasım, 2011.

\item Chen T., Guestrin C., XGBoost: A Scalable Tree Boosting System, 22nd ACM SIGKDD International Conference on Knowledge Discovery and Data Mining, San Francisco-A.B.D., 785-794, 13-17 Ağustos, 2016.

\item Prokhorenkova L., Gusev G., Vorobev A., Dorogush A.V., Gulin A., CatBoost: Unbiased Boosting with Categorical Features, Advances in Neural Information Processing Systems, Montreal-Kanada, 31, 6638-6648, 3-8 Aralık, 2018.

\item Taylor S.J., Letham B., Forecasting at Scale, Am. Stat., 72 (1), 37-45, 2018.

\end{enumerate}

\end{document}


\end{document}
