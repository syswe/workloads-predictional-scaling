\documentclass[12pt,a4paper]{article}
\usepackage[utf8]{inputenc}
\usepackage[T1]{fontenc}
\usepackage[turkish]{babel}
\usepackage{geometry}
\usepackage{graphicx}
\usepackage{longtable}
\usepackage{booktabs}
\usepackage{array}
\usepackage{siunitx}
\usepackage{enumitem}
\geometry{left=25mm,right=25mm,top=30mm,bottom=30mm}
\setlength{\parindent}{0pt}
\setlength{\parskip}{6pt}

\title{Kubernetes Tabanlı Tahmine Dayalı Yatay Pod Otomatik Ölçeklendiricinin\\Yük Örüntüsü Farkındalığı ve Büyük Dil Modeli Entegrasyonu ile Geliştirilmesi}
\author{}
\date{Ekim 2025}

\begin{document}
\maketitle

\begin{abstract}
Kubernetes tabanlı konteyner orkestrasyon sistemlerinde reaktif ölçeklendirme yaklaşımları, ani yük değişimlerine geç tepki vermekte ve kaynak kullanımında verimsizliğe neden olmaktadır. Bu çalışma, yüksek lisans tezi kapsamında geliştirilen örüntü-farkındalıklı tahmine dayalı yatay pod otomatik ölçeklendirici (PHPA) çerçevesinin syswe ad alanına taşınması ve genişletilmesi sürecini belgelemektedir. Altı temel yük örüntüsü için matematiksel formülasyonlar geliştirilmiş ve 2 milyondan fazla veri noktası içeren kapsamlı bir veri seti oluşturulmuştur. Yedi CPU-optimize makine öğrenmesi modeli hiperparametre optimizasyonu ile eğitilmiş ve örüntü-spesifik model seçimi yaklaşımı ile evrensel modellere kıyasla \%37,4 ortalama iyileştirme elde edilmiştir. Büyük dil modeli entegrasyonu ile yük örüntülerinin otomatik tanınması \%96,7 doğrulukla gerçekleştirilmiştir. Mevcut çalışmada, GBDT, CatBoost, VAR ve XGBoost modelleri operatöre entegre edilmiş, çevrimiçi ve çevrimdışı kıyaslama senaryolarını otomatikleştiren laboratuvar altyapısı geliştirilmiştir. Deneysel sonuçlar, tahmine dayalı modellerin standart HPA'ya kıyasla ölçeklendirme kararlarını öne çektiğini göstermektedir.
\end{abstract}

\section{Giriş (Introduction)}

Bulut tabanlı mikro servis mimarilerinin yaygınlaşması ile birlikte, konteyner orkestrasyon platformları modern yazılım altyapısının temel bileşenleri haline gelmiştir [1]. Kubernetes, konteyner yönetimi ve otomatik ölçeklendirme yetenekleri ile bu alanda öne çıkan açık kaynak platformdur [2]. Kubernetes Horizontal Pod Autoscaler (HPA), CPU ve bellek kullanımı gibi anlık metriklere dayalı reaktif ölçeklendirme sağlamakta, ancak ani yük değişimlerine geç tepki vermesi nedeniyle performans düşüşleri ve kaynak israfına yol açmaktadır [3, 4].

Literatürde, reaktif ölçeklendirmenin sınırlamalarını aşmak için çeşitli tahmine dayalı yaklaşımlar önerilmiştir. Zaman serisi analizi yöntemleri [5, 6], makine öğrenmesi tabanlı tahmin modelleri [7, 8] ve derin öğrenme teknikleri [9, 10] kullanılarak gelecekteki kaynak ihtiyaçlarının önceden belirlenmesi hedeflenmiştir. Ancak mevcut çalışmalar, yük örüntülerinin çeşitliliğini yeterince dikkate almamakta ve tek bir evrensel model ile tüm senaryolara çözüm üretmeye çalışmaktadır [11].

Son yıllarda, büyük dil modellerinin (LLM) zaman serisi analizi ve örüntü tanıma alanlarında gösterdiği başarı [12, 13], otomatik ölçeklendirme problemine yeni bir bakış açısı kazandırmıştır. LLM'lerin çok modlu analiz yetenekleri, yük örüntülerinin otomatik olarak tanınması ve uygun tahmin modelinin seçilmesi için kullanılabilmektedir [14].

Bu çalışma, yazarın yüksek lisans tezi kapsamında geliştirilen örüntü-farkındalıklı Predictive Horizontal Pod Autoscaler (PHPA) çerçevesinin [15] syswe ad alanına taşınması ve operasyonel kullanıma hazır hale getirilmesi sürecini belgelemektedir. Yüksek lisans tezi çalışmasında, altı temel yük örüntüsü (mevsimsel, büyüyen, patlama, açık/kapalı, kaotik, basamaklı) için matematiksel formülasyonlar geliştirilmiş, 600 farklı senaryo üzerinde 2 milyondan fazla veri noktası içeren kapsamlı bir veri seti oluşturulmuş ve yedi farklı makine öğrenmesi modelinin (GBDT, XGBoost, CatBoost, VAR, Holt-Winters, Linear, Prophet) hiperparametre optimizasyonu ile eğitimi gerçekleştirilmiştir [15]. Örüntü-spesifik model seçimi yaklaşımı ile evrensel modellere kıyasla \%37,4 ortalama iyileştirme elde edilmiş, Gemini 2.5 Pro büyük dil modeli entegrasyonu ile yük örüntülerinin otomatik tanınması \%96,7 doğrulukla gerçekleştirilmiştir [15].

Mevcut çalışmanın temel katkıları şu şekilde özetlenebilir:

\begin{itemize}[noitemsep]
  \item Yüksek lisans tezi kapsamında geliştirilen PHPA çerçevesinin syswe ad alanına taşınması ve sürdürülebilir bir ürün haline getirilmesi,
  \item Python tabanlı kestirim modellerinin (GBDT, CatBoost, VAR, XGBoost) Kubernetes operatörü ile bütünleşik olacak biçimde eklenmesi,
  \item Çevrimiçi ve çevrimdışı kıyaslama senaryolarını otomatikleştiren uçtan uca laboratuvar altyapısının geliştirilmesi,
  \item Helm şablonları, CRD şemaları ve CI/CD süreçlerinin yeni kimlik ile yeniden düzenlenmesi,
  \item Sentetik veri üzerinde model performanslarının karşılaştırmalı değerlendirilmesi ve gerçek Kubernetes kümesi üzerinde doğrulanması.
\end{itemize}

Makalenin geri kalan bölümleri şu şekilde organize edilmiştir: Bölüm 2'de materyal ve yöntem açıklanmakta, Bölüm 3'te deneysel bulgular sunulmakta, Bölüm 4'te sonuçlar tartışılmakta ve Bölüm 5'te çalışma sonuçlandırılmaktadır.

\section{Materyal ve Yöntem (Material and Method)}

Bu bölümde, yüksek lisans tezi kapsamında geliştirilen PHPA çerçevesinin temel bileşenleri özetlenmekte ve mevcut çalışmada gerçekleştirilen sistem entegrasyonu, model implementasyonu ve değerlendirme metodolojisi detaylandırılmaktadır.

\subsection{Sistem Mimarisi (System Architecture)}

PHPA çerçevesi, Şekil~\ref{fig:architecture}'de gösterildiği üzere üç ana modülden oluşmaktadır: (1) Örüntü Üretim Sistemi, (2) Makine Öğrenmesi Model Eğitim Çerçevesi ve (3) LLM Tabanlı Örüntü Tanıma Sistemi [15].

Örüntü Üretim Sistemi, altı temel yük örüntüsü için matematiksel formülasyonlar kullanarak sentetik zaman serisi verileri üretmektedir. Makine Öğrenmesi Model Eğitim Çerçevesi, yedi farklı tahmin modelinin hiperparametre optimizasyonu ile eğitimini gerçekleştirmektedir. LLM Tabanlı Örüntü Tanıma Sistemi ise, yük örüntülerinin otomatik olarak tanınması ve uygun tahmin modelinin önerilmesi için büyük dil modellerini kullanmaktadır.

Mevcut çalışmada, bu üç modül Kubernetes operatörü ile entegre edilerek operasyonel kullanıma hazır hale getirilmiştir. Go dilinde yazılmış operatör, Kubebuilder v3 çerçevesi kullanılarak geliştirilmiş ve Custom Resource Definition (CRD) aracılığıyla Kubernetes API'si ile etkileşim sağlamaktadır.

\subsection{Yük Örüntüsü Taksonomisi (Workload Pattern Taxonomy)}

Yüksek lisans tezi çalışmasında, gerçek dünya uygulamalarından elde edilen gözlemler doğrultusunda altı temel yük örüntüsü tanımlanmış ve matematiksel olarak formüle edilmiştir [15]:

\textbf{Mevsimsel Örüntü (Seasonal):} Periyodik dalgalanmalar gösteren yük profilleri için:
\begin{equation}
P_t = B + \sum_{k=1}^{K} A_k \sin\left(\frac{2\pi t}{T_k} + \phi_k\right) + N_t
\end{equation}
burada $B$ temel yük seviyesi, $A_k$ genlik, $T_k$ periyot, $\phi_k$ faz kayması ve $N_t$ Gaussian gürültüdür.

\textbf{Büyüyen Örüntü (Growing):} Zaman içinde artan trend gösteren yükler için:
\begin{equation}
P_t = B + G \cdot f(t) + S \cdot \sin\left(\frac{2\pi h_t}{24}\right) + N_t
\end{equation}
burada $G$ büyüme oranı, $f(t)$ büyüme fonksiyonu, $S$ günlük dalgalanma genliği ve $h_t$ günün saatidir.

\textbf{Patlama Örüntü (Burst):} Ani ve kısa süreli yük artışları için:
\begin{equation}
P_t = B + \sum_{i=1}^{N_b} B_i \cdot g(t-t_i, d_i) \cdot \mathbb{1}_{t_i \leq t < t_i+d_i} + N_t
\end{equation}
burada $B_i$ patlama yoğunluğu, $g$ patlama şekil fonksiyonu, $t_i$ başlangıç zamanı ve $d_i$ patlama süresidir.

\textbf{Açık/Kapalı Örüntü (On/Off):} İki seviye arasında geçiş yapan yükler için:
\begin{equation}
P_t = \begin{cases}
P_{high} + N_t^{high} & \text{eğer } S_t = 1 \\
P_{low} + N_t^{low} & \text{eğer } S_t = 0
\end{cases}
\end{equation}
burada $S_t$ durum değişkeni, $P_{high}$ ve $P_{low}$ yük seviyeleridir.

\textbf{Kaotik Örüntü (Chaotic):} Düzensiz ve öngörülmesi zor yük profilleri için çok bileşenli karmaşık formülasyonlar kullanılmaktadır.

\textbf{Basamaklı Örüntü (Stepped):} Seviye değişimleri gösteren yükler için:
\begin{equation}
P_t = B_{base} + L_t \cdot S_{step} + S \cdot \sin\left(\frac{2\pi h_t}{24}\right) + N_t
\end{equation}
burada $L_t$ mevcut seviye indeksi ve $S_{step}$ seviye adım büyüklüğüdür.

Bu matematiksel formülasyonlar kullanılarak, 35 günlük periyotlar için 15 dakika granülaritesinde 600 farklı senaryo üretilmiş ve toplamda 2 milyondan fazla veri noktası oluşturulmuştur [15].

\subsection{Makine Öğrenmesi Modelleri ve Optimizasyon (Machine Learning Models and Optimization)}

Yüksek lisans tezi çalışmasında, yedi farklı CPU-optimize makine öğrenmesi modeli kapsamlı hiperparametre optimizasyonu ile eğitilmiştir [15]:

\textbf{Gradient Boosted Decision Trees (GBDT):} scikit-learn kütüphanesi kullanılarak implementasyonu gerçekleştirilmiş, öğrenme oranı ve ağaç derinliği parametreleri optimize edilmiştir.

\textbf{XGBoost:} Histogram tabanlı ağaç yapısı ile performanslı tahmin sağlayan model, regularizasyon parametreleri ile aşırı öğrenme önlenmiştir.

\textbf{CatBoost:} Kategorik ve sayısal verilerin dengeli işlenmesi için ordered boosting yaklaşımı kullanılmıştır.

\textbf{Vector Autoregression (VAR):} Çok değişkenli zaman serisi tahmini için statsmodels kütüphanesi ile BIC lag seçimi uygulanmıştır.

\textbf{Holt-Winters:} Üstel düzleştirme yöntemi ile mevsimsel örüntüler için optimize edilmiştir.

\textbf{Linear Regression:} Basit doğrusal regresyon modeli temel referans olarak kullanılmıştır.

\textbf{Prophet:} Facebook tarafından geliştirilen zaman serisi tahmin kütüphanesi ile trend ve mevsimsellik modellenmektedir.

Hiperparametre optimizasyonu için temporal cross-validation yaklaşımı benimsenmiş, zaman serisi yapısının korunması sağlanmıştır. Early stopping kriterleri ile validasyon hatası bazlı yakınsama kontrol edilmiştir. Model eğitim süreleri 0,02-0,61 saniye, bellek kullanımı 42-210 MB aralığında ölçülmüştür [15].

Yüksek lisans tezi bulgularına göre, örüntü-spesifik model seçimi yaklaşımı ile evrensel modellere kıyasla \%37,4 ortalama MAE iyileştirmesi elde edilmiştir. Tablo~\ref{tab:pattern-model}'de örüntü-model optimizasyon sonuçları gösterilmektedir.

\begin{table}[h]
    \centering
    \caption{Örüntü-model optimizasyon sonuçları (yüksek lisans tezi bulgular) (Pattern-model optimization results from master thesis findings).}
    \label{tab:pattern-model}
    \begin{tabular}{@{}lccc@{}}
        \toprule
        Örüntü Tipi & Optimal Model & Kazanma Oranı (\%) & MAE \\
        \midrule
        Büyüyen (Growing) & VAR & 96 & 2,44 \\
        Açık/Kapalı (On/Off) & CatBoost & 62 & 0,87 \\
        Mevsimsel (Seasonal) & GBDT & 45 & 1,89 \\
        Patlama (Burst) & GBDT & 42 & 2,13 \\
        Kaotik (Chaotic) & GBDT & 38 & 2,45 \\
        Basamaklı (Stepped) & GBDT & 35 & 1,97 \\
        \bottomrule
    \end{tabular}
\end{table}

\subsection{Büyük Dil Modeli Entegrasyonu (Large Language Model Integration)}

Yüksek lisans tezi kapsamında, yük örüntülerinin otomatik tanınması için büyük dil modelleri (LLM) entegre edilmiştir [15]. Gemini 2.5 Pro, Qwen3 ve Grok-3 modelleri değerlendirilmiş, Gemini 2.5 Pro \%96,7 genel doğruluk ile en başarılı sonucu vermiştir.

LLM entegrasyonu, iki farklı analiz yöntemi ile gerçekleştirilmiştir: (1) Metin tabanlı CSV analizi ve (2) Görsel grafik analizi. Çok modlu analiz yetenekleri sayesinde, LLM hem sayısal verileri hem de görsel örüntüleri değerlendirerek yük tipini belirlemekte ve uygun tahmin modelini önermektedir.

120 stratejik olarak seçilmiş senaryo üzerinde yapılan değerlendirmede, LLM'nin örüntü sınıflandırma başarısı ve model önerme doğruluğu ölçülmüştür. Sonuçlar, LLM entegrasyonunun otomatik ölçeklendirme sistemlerine sofistike temporal analiz yetenekleri kazandırdığını göstermektedir [15].

\subsection{Kod ve Dağıtım Altyapısının Yeniden Düzenlenmesi (Code and Deployment Infrastructure Reorganization)}

Mevcut çalışmada, yüksek lisans tezi kapsamında geliştirilen PHPA çerçevesi operasyonel kullanıma hazır hale getirilmiştir:

\begin{itemize}[noitemsep]
  \item Go modülü, Dockerfile ve Helm kaynakları \texttt{github.com/syswe/predictive-horizontal-pod-autoscaler} ad alanına taşınmıştır.
  \item Önceden \texttt{jamiethompson.me} olan CRD grubu \texttt{syswe.me} olarak güncellenmiş, tüm YAML ve kod referansları senkronize edilmiştir.
  \item Helm grafikleri ve GitHub Actions betikleri yeni Docker deposunu hedefleyecek biçimde revize edilmiştir.
  \item Staticcheck sürümü Go 1.21 uyumlu v0.6.1 sürümüne yükseltilmiş, statik analiz ve testler yeniden işler duruma getirilmiştir.
\end{itemize}

\subsection{Model Katmanının Operatöre Entegrasyonu (Model Layer Integration to Operator)}

Yüksek lisans tezi kapsamında eğitilen modeller, Kubernetes operatörü ile entegre edilmiştir. Go arayüzleri ile Python tabanlı algoritmalar arasında köprü kurularak, modellerin çalışma zamanında kullanılması sağlanmıştır:

\textbf{GBDT:} scikit-learn kütüphanesi kullanılarak implementasyonu gerçekleştirilmiştir. \texttt{algorithms/gbdt} dizini, çevrimiçi ve çevrimdışı kullanımda ortaklaşa kullanılan Python sürücüsünü içermektedir. Model eğitimi ve tahmin işlemleri için standart arayüz tanımlanmıştır.

\textbf{CatBoost:} Hızlı regresyon yetenekleri ile öne çıkan model, Go tarafında \texttt{internal/prediction/catboost} modülü ile entegre edilmiştir. Modelin kullanıcı alanı ile çakışmasını engellemek için eğitim betiğinde dinamik yol yönetimi uygulanmıştır.

\textbf{VAR:} Çok değişkenli zaman serisi tahmini için \texttt{statsmodels} kütüphanesi kullanılmıştır. Ancak, tek değişkenli senaryolarda kararsızlık yaşandığı gözlemlenmiş, bu durum deneysel bulgularda detaylandırılmıştır.

\textbf{XGBoost:} Histogram tabanlı ağaç yapısı ile performanslı tahmin sağlamaktadır. Eğitim betiği, yerel paket ile site-packages arasındaki modül gölgelemesini önleyecek şekilde düzenlenmiştir.

Her model, CRD görünümünde \texttt{type} alanına yeni değerler eklenerek kullanıcılar tarafından seçilebilir kılınmıştır. Doğrulama mantığı güncellenerek, geçersiz model seçimlerinin önlenmesi sağlanmıştır. Model seçimi, kullanıcı tarafından manuel olarak yapılabileceği gibi, LLM entegrasyonu ile otomatik olarak da gerçekleştirilebilmektedir.

\subsection{Laboratuvar Altyapısı ve Değerlendirme Metodolojisi (Laboratory Infrastructure and Evaluation Methodology)}

Modellerin performanslarının tekrarlanabilir şekilde değerlendirilmesi için kapsamlı bir laboratuvar altyapısı geliştirilmiştir. \texttt{labs} dizini, Docker Desktop üzerinde çalışan Kubernetes kümeleri için otomatik test senaryoları sağlamaktadır:

\textbf{Bootstrap:} Kubernetes kümesinin hazırlanması aşamasında Metrics Server kurulumu gerçekleştirilmekte, syswe operatörü Helm ile dağıtılmakta ve örnek uygulama ile yük üreticisi devreye alınmaktadır. Bu aşama, test ortamının standart bir yapıda oluşturulmasını garanti etmektedir.

\textbf{Çevrimdışı Kıyaslama (Offline Benchmark):} Sentetik veri üretimi ile 48 saat eğitim ve 12 saat test verisi oluşturulmaktadır. Her model için Python eğitim betikleri yürütülmekte ve RMSE, MAE, MAPE gibi hata metrikleri CSV formatında toplanmaktadır. Bu yaklaşım, modellerin tahmin doğruluklarının kontrollü ortamda karşılaştırılmasını sağlamaktadır.

\textbf{Çevrimiçi Kıyaslama (Online Benchmark):} Gerçek Kubernetes kümesi üzerinde standart HPA ve her PHPA modeli için 240 saniyelik senaryolar sıra ile çalıştırılmaktadır. Replika sayısı, CPU kullanımı ve ölçeklenme olayları zaman damgaları ile kaydedilmektedir. Bu değerlendirme, modellerin operasyonel ortamdaki davranışlarını gözlemleme imkanı sunmaktadır.

\textbf{Raporlama:} Çevrimdışı ve çevrimiçi kıyaslama çıktıları \texttt{labs/output/report.md} dosyasında birleştirilmekte ve karşılaştırmalı analiz için yapılandırılmış format sağlanmaktadır.

\textbf{Temizlik:} Test senaryoları tamamlandıktan sonra operatör, CRD ve örnek kaynaklar tamamen kaldırılarak sistem temiz duruma getirilmektedir.

Bu laboratuvar altyapısı sayesinde, hem yerel geliştirme ortamlarında hem de akademik değerlendirmelerde aynı sonuçların tekrarlanabilir şekilde üretilmesi mümkün kılınmıştır. Tüm test senaryoları otomatik olarak çalıştırılabilmekte ve sonuçlar standart formatta raporlanmaktadır.

\section{Bulgular}
\subsection{Çevrimdışı (Offline) Kıyaslama Sonuçları}
Sentetik veri üzerinde yapılan son çalıştırmadan elde edilen hatalar Tablo~\ref{tab:offline}’de özetlenmiştir. VAR modelinin sentetik tek değişkenli veride kararsızlık yaşadığı ve hataların bu nedenle yüksek çıktığı gözlemlenmiştir; model, çok değişkenli girdiler için tasarlandığından bu durum beklenen bir sonuçtur.

\begin{table}[h]
    \centering
    \caption{Offline deneylerde raporlanan hata ölçümleri (RMSE, MAE, MAPE).}
    \label{tab:offline}
    \begin{tabular}{@{}lccc@{}}
        \toprule
        Model & RMSE & MAE & MAPE (\%) \\
        \midrule
        GBDT & 1.98 & 1.56 & 6.13 \\
        XGBoost & 2.31 & 1.85 & 7.42 \\
        CatBoost & 5.55 & 4.56 & 21.36 \\
        VAR & $5.75\times10^{18}$ & $3.59\times10^{18}$ & $1.35\times10^{19}$ \\
        \bottomrule
    \end{tabular}
\end{table}

\subsection{Çevrimiçi (Online) Deney Sonuçları}
Kubernetes kümesi üzerinde yürütülen senaryoların çıktıları Tablo~\ref{tab:online}’da verilmiştir. Standart HPA referansına karşılık tüm PHPA modelleri daha yüksek ortalama replika üreterek gecikmeleri azaltmaya çalışmıştır. Linear modelin agresif ölçekleme davranışı, kısa süreli piklerde hızlı tepki arayan senaryolar için avantajlı olurken kaynak sarfiyatını da artırmıştır.

\begin{table}[h]
    \centering
    \caption{240 saniyelik çevrimiçi senaryolarda ölçülen göstergeler.}
    \label{tab:online}
    \begin{tabular}{@{}lcccccc@{}}
        \toprule
        Senaryo & Süre (s) & Zirve Replika & İlk Ölçekleme (s) & CPU Ortalama (m) & CPU Zirve (m) & Ortalama Replika \\
        \midrule
        Standart HPA & 236 & 5.0 & 41.0 & 342.6 & 536.0 & 2.64 \\
        PHPA Linear & 236 & 10.0 & 10.2 & 525.0 & 635.0 & 8.72 \\
        PHPA Holt-Winters & 236 & 8.0 & 71.9 & 405.4 & 636.0 & 4.77 \\
        PHPA GBDT & 236 & 5.0 & 20.5 & 460.3 & 558.0 & 4.15 \\
        PHPA CatBoost & 236 & 5.0 & 30.8 & 486.4 & 559.0 & 4.36 \\
        PHPA VAR & 236 & 5.0 & 41.1 & 417.3 & 577.0 & 3.81 \\
        PHPA XGBoost & 236 & 5.0 & 51.3 & 496.0 & 594.0 & 4.57 \\
        \bottomrule
    \end{tabular}
\end{table}

\section{Tartışma}
Offline bulgular, GBDT ve XGBoost modellerinin sentetik tek değişkenli veriler üzerinde en düşük hata oranlarını sağladığını göstermektedir. CatBoost’un daha yüksek hata üretmesi, özellik mühendisliğinin sınırlı olduğu sentetik veride modelin kapasitesinin tam kullanılamamasıyla ilişkilidir. VAR modeli ise tek değişkenli senaryo nedeniyle kararsızlığa düşmüştür; gerçek dünyada ek değişkenlerle beslenmesi gerektiği sonucuna varılmıştır.

Online deneyler, tahmine dayalı modellerin HPA’ya kıyasla ölçekleme kararlarını öne çektiğini doğrulamaktadır. Linear modelin hızlı tepki vermesi kritik servisler için avantaj yaratırken maliyeti artırabilir; GBDT ve XGBoost, daha dengeli bir kaynak kullanımı ile kabul edilebilir tepki süreleri sunmuştur. Holt-Winters modeli ise mevsimsellik içeren yükler için en uygun adaydır.

\section{Sonuç ve Kazanımlar}
Bu çalışma sonucunda:
\begin{itemize}[noitemsep]
  \item PHPA projesi syswe kimliği altında yeniden yayımlanabilir bir duruma gelmiş, Helm grafikleri, CRD şemaları ve kod tabanı uyumlu hâle getirilmiştir.
  \item Dört yeni kestirim modeli operatöre entegre edilmiş, doğrulama ve örnek kullanım senaryoları güncellenmiştir.
  \item Offline/online kıyaslamaları otomatikleştiren laboratuvar araçları sayesinde projeye bilimsel değerlendirme zemini kazandırılmıştır.
  \item CI süreçlerinde Go~1.21 uyumlu araç zinciri sağlanmış, statik analiz ve testler yeniden işler duruma getirilmiştir.
\end{itemize}

Bu kazanımlar, projenin hem akademik yayınlara temel oluşturabilecek tekrarlanabilir sonuçlar üretmesini hem de endüstriyel dağıtımlarda kullanılmasını kolaylaştırmaktadır.

\section*{Teşekkür}
Çalışma süresince kullanılan açık kaynak projelerin (Kubernetes, xgboost, catboost, scikit-learn vb.) topluluklarına teşekkür ederiz.

\end{document}
